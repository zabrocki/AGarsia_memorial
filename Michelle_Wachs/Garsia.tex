\documentclass[12pt]{article}
\usepackage{amssymb}

\def\Z{{\mathbb{Z}}}
\def\R{{\mathbb{R}}}
\def\Q{{\mathbb{Q}}}
\def\adj{\rm adj}
\usepackage{bbm}
\usepackage{tikz}
%\usepackage{verbatim}
\usepackage{pgf}
\usepackage{graphicx}

\thispagestyle{empty}
\usepackage{wrapfig}
\usepackage{graphics}


\begin{document}
\begin{center} Tribute to Adriano Garsia  by Michelle Wachs
\end{center}

\vspace{.2in}
I was one of Adriano’s first combinatorics students at UCSD in the mid-1970s. I met him during my second year of graduate school when I took his Real Analysis course. He was an extraordinary teacher, and since he was also renowned as a leading researcher in analysis, I hoped he might become my advisor in that field.  Before I even got up the courage to ask him, he invited me to his office to share  ideas he had about simplifying the proof of the 
Hu--Tucker algorithm for optimal binary search trees. I was surprised: this wasn’t real analysis! In fact, it was combinatorics, though I didn’t know what that was at the time. I also didn’t know that Adriano was transitioning from analysis to combinatorics, where he would become one of the world's foremost leaders in algebraic combinatorics for almost 50 years.

I worked on Adriano's binary search tree problem and had a breakthrough:  his ideas led, not to a simpler proof of the Hu--Tucker algorithm, but  to a  new simpler algorithm for constructing optimal binary search trees. 
I was so excited to show him what I had done, I went to the university and sat on the floor outside his office door for what felt like hours waiting  for him to arrive. (This was long before email, cell phones, or texting.)  Little did I know, he was at his favorite place -- the beach -- which also happened to be my favorite place -- what luck!   

\begin{wrapfigure}{r}{0.3\textwidth}
    \centering
    \includegraphics[width=0.95\linewidth]{IMG_4082.jpeg} 
 {\footnotesize Adriano, Diane, and the Boogie Board at Garsiafest, Scripps, 2019}
    \end{wrapfigure}
The beach was Adriano's second office. 
His students, collaborators, and many distinguished visitors would find him at his  encampment on the beach (first in Del Mar, later at La Jolla Shores) where he would share his beautiful mathematical ideas and listen to theirs. He also shared his passion for riding the waves.    He got a Morey Boogie Board when they first came out in 1975, and after seeing his, I went right to the store to get one too. In the years to come he would bring a stack of them to the beach for his visitors to use.  At his 91st birthday conference — held appropriately at the Scripps Institute of Oceanography overlooking the beach — a boogie board signed by all the participants was presented to him.  The NSF program director remarked in his speech at the conference that  Adriano, at $91$, was the oldest active principal investigator on an NSF grant in any scientific field. 


\begin{wrapfigure}{r}{0.4\textwidth}
    \centering
    \includegraphics[width=0.95\linewidth]{cmposets.pdf} 
  {\footnotesize Adriano, Richard, Anders at Banff Conference on Ordered Sets, 1981}
    \end{wrapfigure}
 From the mid-1970s to the late-1980s, Adriano's pioneering work helped shape the field of algebraic combinatorics and made him  one of the field's true founding leaders. So many of the ideas he introduced during that  period continue to influence the field today. I’d like to share just a few threads from that time that have had a deep and lasting impact on my own work.
 Adriano’s early research on 
q-analogs and symmetric functions — much of it done in collaboration with Ira Gessel and Jeff Remmel — laid the groundwork for his later celebrated work with Mark Haiman on Macdonald polynomials and diagonal harmonics. Those early papers -- as well as his subsequent papers --  were also a key source of inspiration for my own research on 
q-analogs and symmetric functions throughout my career. His work on Cohen-Macaulay posets, including his influential survey chapter  with Anders Bj\"orner and Richard Stanley, played a central role in the early development of topological combinatorics,  a field I’ve been fortunate to be part of since  early in my career. And Adriano’s beautiful work on the Lie representation of the symmetric group continues to resonate with me. It sparked ideas that led to some of my own early research and has remained a source of insight and motivation in my work on this topic ever since.
 






Adriano was an exceptional advisor -- extremely generous with his time and ideas. He had a rare gift for recognizing and nurturing the individual talents and abilities of his students.   Many of us feel we owe what ever success we have achieved to his mentorship.  In addition to sharing his great ideas, he was supportive and passionate.  His excitement about our work was a great confidence booster.  He and I often had spirited debates when discussing math,  but we always resolved things amicably.   The good thing was that I always felt free to express myself openly and honestly. 


Adriano was there for his students long after they graduated.    For many years after completing my Ph.D., my husband (a fellow UCSD Ph.D. -- in differential geometry)  and I would spend the summers visiting UCSD, and we even spent a full academic year there as visiting associate professors.   
 Adriano       helped make this possible by being a wonderful mathematical host and finding  places for us to stay.   
 During those visits, there was a vibrant atmosphere in algebraic combinatorics, with seminars given
 \begin{wrapfigure}{r}{0.3\textwidth}
 \centering
    \includegraphics[width=0.95\linewidth]{IMG_2458.jpg} 
  {\footnotesize UCSD Combinatorics Seminar, Del Mar, 1981,
Adriano, Richard, me, Dennis, Jeff}
    \end{wrapfigure}
  by distinguished visitors such as Richard Stanley, Anders Bj\"orner, Alain Lascoux, and Dennis Stanton, to name a few. 
 During my first summer visit, Adriano suggested to Anders that he discuss with me his problem  on shellability of Bruhat order.   This led to a lifelong collaboration and friendship with Anders for which I remain deeply grateful to Adriano. This was typical Adriano—he fostered many enduring mathematical relationships.

My husband Greg and I took some memorable trips with Adriano and his wife Diane, both mathematical and personal.  We had a great time in Rome with Adriano teaching us all how to behave in Italy and how to sniff out good restaurants. Our young children, Brian and Gabriela, were inseparable during our stays in 1992 at Mittag-Leffler and in 1994 at Garsiafest in Taormina. Whether at home or away,  Adriano loved to entertain his colleagues and friends by treating us to unforgettable Italian and Tunisian meals, which he cooked himself.

\begin{center}
 \includegraphics[height=3in]{EmbeddedImage.jpeg} 
 \\ {\footnotesize Garsiafest, Taormina, 1994}
 \end{center}




To so many of us, Adriano was more than a mentor, colleague, or friend—he was family.  Once, during a short visit to San Diego, he and Diane invited me to stay with them. At the time, their two-bedroom country home  was full, with Diane’s brother occupying the second room.  So my “room” was the front porch.   I left my suitcase in the living room while I slept out on the porch.  My suitcase was a  beat-up leather suitcase with a handle that had become partially detached.  When I came in the next morning, to my surprise, Adriano  was busy sewing the handle back on with a giant needle and thick thread. I was so touched. What could be sweeter and more fatherly than that?  



Adriano’s passion for mathematics—and for life—was legendary.  His work continues to have a strong influence on my work and that of so many others.  What a profound privilege it has been to have known this remarkable man for so many years. 

\end{document}
